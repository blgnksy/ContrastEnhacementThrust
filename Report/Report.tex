% !TEX TS-program = pdflatex
% !TEX encoding = UTF-8 Unicode

% This is a simple template for a LaTeX document using the "article" class.
% See "book", "report", "letter" for other types of document.

\documentclass[11pt]{article} % use larger type; default would be 10pt

\usepackage[utf8]{inputenc} % set input encoding (not needed with XeLaTeX)
\usepackage[american]{babel}
%%% Examples of Article customizations
% These packages are optional, depending whether you want the features they provide.
% See the LaTeX Companion or other references for full information.

%%% PAGE DIMENSIONS
\usepackage[margin=2cm,left=2cm,includefoot]{geometry}
\geometry{a4paper} % or letterpaper (US) or a5paper or....
% \geometry{margin=2in} % for example, change the margins to 2 inches all round
% \geometry{landscape} % set up the page for landscape
%   read geometry.pdf for detailed page layout information

\usepackage{graphicx} % support the \includegraphics command and options

% \usepackage[parfill]{parskip} % Activate to begin paragraphs with an empty line rather than an indent

%%% PACKAGES
\usepackage{booktabs} % for much better looking tables
\usepackage{array} % for better arrays (eg matrices) in maths
\usepackage{paralist} % very flexible & customisable lists (eg. enumerate/itemize, etc.)
\usepackage{verbatim} % adds environment for commenting out blocks of text & for better verbatim
\usepackage{subfig} % make it possible to include more than one captioned figure/table in a single float
\usepackage[hidelinks]{hyperref} %clickable references
\usepackage{float}%float position
\usepackage[document]{ragged2e} %justify
\usepackage{amsmath} %multiline equations
\usepackage{listings}
\usepackage{color}
\usepackage{cleveref}
\newcommand{\crefrangeconjunction}{ to~}
% These packages are all incorporated in the memoir class to one degree or another...
% for code samples package listings
\lstset{language=C,
  frame=single,
  breaklines=true,
  basicstyle=\ttfamily,
  keywordstyle=\color{blue}\ttfamily,
  stringstyle=\color{red}\ttfamily,
  commentstyle=\color{green}\ttfamily,
  morecomment=[l][\color{magenta}]{\#},
  morekeywords={Npp8u,__host__,__device__,thrust,reduce,for_each,transform,minimum,maximum,placeholders}
}

%%% HEADERS & FOOTERS
\usepackage{fancyhdr} % This should be set AFTER setting up the page geometry
\pagestyle{fancy} % options: empty , plain , fancy
\renewcommand{\headrulewidth}{0pt} % customise the layout...
\lhead{}\chead{}\rhead{}
\lfoot{}\cfoot{\thepage}\rfoot{}

%%% SECTION TITLE APPEARANCE
\usepackage{sectsty}
\allsectionsfont{\sffamily\mdseries\upshape} % (See the fntguide.pdf for font help)
% (This matches ConTeXt defaults)

%%% ToC (table of contents) APPEARANCE
\usepackage[nottoc,notlof,notlot]{tocbibind} % Put the bibliography in the ToC
\usepackage[titles,subfigure]{tocloft} % Alter the style of the Table of Contents
\renewcommand{\cftsecfont}{\rmfamily\mdseries\upshape}
\renewcommand{\cftsecpagefont}{\rmfamily\mdseries\upshape} % No bold!

%%% END Article customizations

%%% The "real" document content comes below...

%\date{} % Activate to display a given date or no date (if empty),
         % otherwise the current date is printed 


%opening
\title{}
\author{}

\begin{document}
% Title Page
\begin{titlepage}
	\begin{center}
		\line(10,0){400}\\
		[4mm] %for add spacing
		\huge{\bfseries Contrast Enhancement} \\
		\huge{\bfseries Using Thrust Library} \\
		[1mm]
		\includegraphics[width=4cm]{./Figures/Thrust.jpg}
		\line(10,0){400}\\
		[1 cm]
		\textsc{\LARGE Bilgin Aksoy}\\
		[1 cm]
		\textsc{\large MMI713-Applied Parallel Programming}\\
		[10 cm]
	\end{center}
	
	\begin{flushright}
		\textsc{\large Bilgin Aksoy\\
		MMI\\
		2252286\\
		01 January 2018\\
		}
	\end{flushright}
\end{titlepage} 
\pagenumbering{arabic}
\setcounter{page}{1}

\section{Problem Definition }
	\justifying During the assignment, the contrast enhancement algorithm were developed using the THRUST library.
	\begin{itemize}
		\item Finding the minimum valued pixel. $(nMin)$
		\item Finding the maximum valued pixel. $(nMax)$
		\item Applying the following operation(Equ-\ref{equ:ce}) to all pixels $(i,j)$ of the source $(pSrc)$ and write to the destination $(pDst)$.
		\begin{equation}
		\label{equ:ce}
		pDst(i,j)=\frac {pSrc(i,j)-nMin}{nMax-nMin}\times 255
		\end{equation}
	\end{itemize}
\section{Algorithm Desciption}
	\justifying I used 4 Thrust function to be able to find the expected outcome:
	\subsubsection{Finding Minimum-Maximum} Thrust library provides reduction algorithm. I used the reduce function to find the minimum and the maximum.(Listing-\ref{lst:minmax})  I realized that the reduction module is also run the kernel two times, when I used the NSight Performance Analysis tool. In the previous implementation (for Assignment-II) I also used the each kernel two times.
		\begin{lstlisting}[caption={Functor},label={lst:minmax}]
int nMin = thrust::reduce(pDst_Dev.begin(), pDst_Dev.end(),257, thrust::minimum<int>());
int nMax = thrust::reduce(pDst_Dev.begin(), pDst_Dev.end(), 0, thrust::maximum<int>());
		\end{lstlisting}
\hfill
	\subsubsection{Subtract The Minimum} I used the Thrust iterator algorithm to subtract the minimum value from the input.(Listing-\ref{lst:subtract})
		\begin{lstlisting}[caption={Functor},label={lst:subtract}]
thrust::for_each(pDst_Dev.begin(), pDst_Dev.end(), thrust::placeholders::_1 -= nMin);
		\end{lstlisting}
		\hfill	
	\subsubsection{Multiplication-Division}  I used the Thrust transform module and send each pixel to a functor(Listing-\ref{lst:muldiv}). This functor calculates the multiplication and division operations.
		\begin{lstlisting}[caption={Functor},label={lst:muldiv}]
struct muldiv_functor
{
	unsigned int a;

	muldiv_functor(unsigned int nConstant, unsigned int nNormalizer) {
		a = round(nConstant / nNormalizer); 
	}

	__host__ __device__
	Npp8u operator()(const Npp8u& x) const
	{
	return a*x ;
	}
};
		\end{lstlisting}
		\hfill
\section{Benchmarking}

	\justifying  The algorithm implemented on CPU is still faster (Table-1) than both the algorithm using NPP Library, GPU implementation(for Assignment-II), and new implementation (for Assignment-III) several times. One of the main bottleneck in parallel programming is memory copies between host and
	device. Table-2 shows the durations for copying the data. I compared the results of previous implementation(for Assignment-II), and new implementation. The input and results can be seen on \Cref{fig:before,fig:GPU,fig:Thrust}.  The output is visually good.   \\
	
	\begin{table}[H] % H stands for here not anywhere else
		\centering		
		\caption[The Time-Consuming Of The Three Algorithms]{ The Time-Consuming Of The Three Algorithms }
		\label{tab:table_1}	
		\begin{tabular}{l c c c}
			Algorithm & Minimum ($\mu s$) & Maximum ($\mu s$) & Average Time ($\mu s$) \\ \hline
			CPU &	15,3902	&	23,0447	&	19,01675	 \\ 
			GPU & 	654,743	&	779,464	&	703,4921	\\ 
			NPP & 	784,412	&	888,618	&	833,2274   \\  
			Thrust&	1110,455&	1121,249&	6893,76 	\\ 
		\end{tabular}
	\end{table}
\hfill
	% Table generated by Excel2LaTeX from sheet 'Worksheet1'
	\begin{table}[htbp]
		\centering
		\caption{Time Consuming-Memory Copy}
		\label{tab:table_2}%		
		\begin{tabular}{llc c}
			Source & Destination & \multicolumn{1}{l}{Duration ($\mu s$)} & \multicolumn{1}{l}{Size (bytes)} \\\hline  
			HostUnpinned & Device & 21123 & 262144 \\  	
			Device & HostUnpinned & 800  & 4 \\
			Device & HostUnpinned & 832   & 4 \\
			Device & HostUnpinned & 20483 & 262144 \\
		\end{tabular}%
	\end{table}%
\hfill
	\begin{figure}
		\centering
		\subfloat[\label{fig:before}Before Enhancement]{\includegraphics[scale=0.35]{./Figures/lena_before.png}}
		\hfill
		\subfloat[\label{fig:GPU}GPU Result]{\includegraphics[scale=0.35]{./Figures/lena_after_GPUs.png}}
		\hfill
		\subfloat[\label{fig:Thrust}Thrust Result]{\includegraphics[scale=0.35]{./Figures/lena_after_GPU_Thrust.png}}
		\caption{Input \ref{fig:before}, GPU Result \ref{fig:GPU}, and Thrust Result \ref{fig:Thrust} }
	\end{figure}
	
\section{Pros-Cons of Solution}
	\justifying Using Thrust library solves many problems about CUDA programming. One doesn't manually manage memory allocations, copy operations, kernel execution parameters etc., and  can easily implement an parallel algorithm.  \\
	
\section{Discussion}
	\justifying The one of the main reasons for CPU algorithm is faster than GPU is copy operation from host to device. Another reason is the input size is very small, thinking the device capabilities. My CUDA device has 1920 cores, many of which didn't used at all. Using Thrust library is very easy.  But I couldn't be so comfortable when thinking Thrust library like a black-box. Especially, the kernel execution parameters \\
	
\section{Environment}
	% Table generated by Excel2LaTeX from sheet 'Worksheet1'
	\begin{table}[!htbp]
		\centering
		\caption{Environment}		
		\label{tab:addlabel}%
		\begin{tabular}{ll}
			Properties & Specifications \\\hline  
			GPU Name & GeForce GTX 1070 \\  
			Driver Type & WDDM \\  
			PCI Bandwidth (GB/s) & 15,754 \\  
			Frame Buffer Physical Size (MiB) & 8192 \\  
			Frame Buffer Bus Width (bits) & 256 \\  
			RAM Type & GDDR5 \\  
			Frame Buffer Bandwidth (GB/s) & 256,256 \\  
			Graphics Clock (MHz) & 1746,5 \\  
			Processor Clock (MHz) & 1746,5 \\  
			Memory Clock (MHz) & 4004 \\  
			SM Count & 15 \\  
			CUDA Cores & 1920 \\  
		\end{tabular}%
	\end{table}%

\end{document}