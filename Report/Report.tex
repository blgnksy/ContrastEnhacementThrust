% !TEX TS-program = pdflatex
% !TEX encoding = UTF-8 Unicode

% This is a simple template for a LaTeX document using the "article" class.
% See "book", "report", "letter" for other types of document.

\documentclass[11pt]{article} % use larger type; default would be 10pt

\usepackage[utf8]{inputenc} % set input encoding (not needed with XeLaTeX)

%%% Examples of Article customizations
% These packages are optional, depending whether you want the features they provide.
% See the LaTeX Companion or other references for full information.

%%% PAGE DIMENSIONS
\usepackage[margin=2cm,left=2cm,includefoot]{geometry}
\geometry{a4paper} % or letterpaper (US) or a5paper or....
% \geometry{margin=2in} % for example, change the margins to 2 inches all round
% \geometry{landscape} % set up the page for landscape
%   read geometry.pdf for detailed page layout information

\usepackage{graphicx} % support the \includegraphics command and options

% \usepackage[parfill]{parskip} % Activate to begin paragraphs with an empty line rather than an indent

%%% PACKAGES
\usepackage{booktabs} % for much better looking tables
\usepackage{array} % for better arrays (eg matrices) in maths
\usepackage{paralist} % very flexible & customisable lists (eg. enumerate/itemize, etc.)
\usepackage{verbatim} % adds environment for commenting out blocks of text & for better verbatim
\usepackage{subfig} % make it possible to include more than one captioned figure/table in a single float
\usepackage[hidelinks]{hyperref} %clickable references
\usepackage{float}%float position
\usepackage[document]{ragged2e} %justify
\usepackage{amsmath} %multiline equations
\usepackage{listings}
\usepackage{color}
\usepackage{cleveref}
\newcommand{\crefrangeconjunction}{ to~}
% These packages are all incorporated in the memoir class to one degree or another...
% for code samples package listings
\lstset{frame=tb,
  breaklines=true,
  basicstyle=\ttfamily,
  keywordstyle=\color{blue}\ttfamily,
  stringstyle=\color{red}\ttfamily,
  commentstyle=\color{green}\ttfamily,
  morecomment=[l][\color{magenta}]{\#}
}

%%% HEADERS & FOOTERS
\usepackage{fancyhdr} % This should be set AFTER setting up the page geometry
\pagestyle{fancy} % options: empty , plain , fancy
\renewcommand{\headrulewidth}{0pt} % customise the layout...
\lhead{}\chead{}\rhead{}
\lfoot{}\cfoot{\thepage}\rfoot{}

%%% SECTION TITLE APPEARANCE
\usepackage{sectsty}
\allsectionsfont{\sffamily\mdseries\upshape} % (See the fntguide.pdf for font help)
% (This matches ConTeXt defaults)

%%% ToC (table of contents) APPEARANCE
\usepackage[nottoc,notlof,notlot]{tocbibind} % Put the bibliography in the ToC
\usepackage[titles,subfigure]{tocloft} % Alter the style of the Table of Contents
\renewcommand{\cftsecfont}{\rmfamily\mdseries\upshape}
\renewcommand{\cftsecpagefont}{\rmfamily\mdseries\upshape} % No bold!

%%% END Article customizations

%%% The "real" document content comes below...

%\date{} % Activate to display a given date or no date (if empty),
         % otherwise the current date is printed 


%opening
\title{}
\author{}

\begin{document}
% Title Page
\begin{titlepage}
	\begin{center}
		\line(10,0){400}\\
		[4mm] %for add spacing
		\huge{\bfseries Contrast Enhancement} \\
		\huge{\bfseries Using Thrust Library} \\
		[1mm]
		\line(10,0){400}\\
		[1 cm]
		\textsc{\LARGE Bilgin Aksoy}\\
		[1 cm]
		\textsc{\large MMI713-Applied Parallel Programming}\\
		[10 cm]
	\end{center}
	
	\begin{flushright}
		\textsc{\large Bilgin Aksoy\\
		MMI\\
		2252286\\
		01 January 2018\\
		}
	\end{flushright}
\end{titlepage} 
\pagenumbering{arabic}
\setcounter{page}{1}

\section{Problem Definition }
	\justifying During the assignment, the contrast enhancement algorithm were developed using the THRUST libary.
\section{Algorithm Desciption}
	\justifying I implemented 3 function(for CPU version) and 4 kernel(for GPUversion). \\
	\subsection{GPU Algorithm}
		\justifying I used 4 Thrust function to be able to find the expected outcome:
		\subsubsection{Finding Minimum} fsdfsdf 
		\subsubsection{Finding Maximum} fsdfsdf 	
		\subsubsection{Subtract Minimum} dasdasd. 
		\subsubsection{Multiply}  dasdsdsda. 
\section{Benchmarking}

	\justifying  sdasda \\
	
	\begin{table}[H] % H stands for here not anywhere else	
		\caption[The Time-Consuming Of The Three Algorithms]{ The Time-Consuming Of The Three Algorithms }
		\label{tab:table_1}	
		\begin{tabular}{l c c c}
			Algorithm & Minimum ($\mu s$) & Maximum ($\mu s$) & Average Time ($\mu s$) \\ \hline\hline 
			CPU &	15,3902	&	23,0447	&	19,01675 \\ \hline
			GPU & 	654,743	&	779,464	&	703,4921\\ \hline
			NPP & 	784,412	&	888,618	&	833,2274   \\  \hline
		\end{tabular}
		\centering	

	\end{table}

	% Table generated by Excel2LaTeX from sheet 'Worksheet1'
	\begin{table}[htbp]
		\centering
		\caption{Time Consuming-Memory Copy}
		\label{tab:table_2}%		
		\begin{tabular}{llc c}
			Source & Destination & \multicolumn{1}{l}{Duration ($\mu s$)} & \multicolumn{1}{l}{Size (bytes)} \\\hline \hline 
			HostUnpinned & Device & 21251 & 262144 \\ \hline
			Device & HostUnpinned & 1121  & 512 \\\hline
			Device & HostUnpinned & 576   & 512 \\\hline
			Device & HostUnpinned & 20067 & 262144 \\\hline
		\end{tabular}%
	\end{table}%

	% Table generated by Excel2LaTeX from sheet 'Worksheet1'
	\begin{table}[htbp]
		\centering		
		\caption{Kernel Execution Time and Achieved Occupancy}
		\label{tab:table_3}%
		\begin{tabular}{lcc}
			Function Name & Duration ($\mu s$) & Achieved Occupancy \\ \hline \hline 
			MinimumKernel & 73,824 & 0,69 \\ \hline
			MaximumKernel & 74,656 & 0,69 \\ \hline
			MinimumKernel & 13,152 & 0,01 \\ \hline
			MaximumKernel & 12,96 & 0,01 \\ \hline
			SubtractKernel & 22,464 & 0,78 \\ \hline
			MultiplyKernel & 73,696 & 0,82 \\ \hline
		\end{tabular}%
		\caption{Kernel Execution Time and Achieved Occupancy}
		\label{tab:table_3}%
	\end{table}%
	\begin{figure}
		\centering
		%\subfloat[\label{fig:before}Before Enhancement]{\includegraphics[scale=0.35]{./Figures/lena_before.png}}
		\hfill
		%\subfloat[\label{fig:CPU}CPU Result]{\includegraphics[scale=0.35]{./Figures/lena_afterCPU3.png}}
		\hfill
		%\subfloat[\label{fig:GPU}GPU Result]{\includegraphics[scale=0.35]{./Figures/lena_after_GPUs.png}}
		\caption{Input \ref{fig:before}, CPU Result-\ref{fig:CPU}, and GPU Result \ref{fig:GPU} }
	\end{figure}
	
\section{Pros-Cons of Solution }
	\justifying The algorithm uses the device efficiently. But using minimum and maximum kernel second time is the main cons of the solution.
\section{Discussion}
	\justifying The one of the main reasons for CPU algorithm is faster than GPU is copy operation. The other reason is the input size is small.\\
\section{Environment}
	% Table generated by Excel2LaTeX from sheet 'Worksheet1'
	\begin{table}[htbp]
		\centering
		\caption{Add caption}		
		\label{tab:addlabel}%
		\begin{tabular}{ll}
			Properties & Specifications \\\hline \hline 
			GPU Name & GeForce GTX 1070 \\ \hline 
			Driver Type & WDDM \\ \hline 
			PCI Bandwidth (GB/s) & 15,754 \\ \hline 
			Frame Buffer Physical Size (MiB) & 8192 \\ \hline 
			Frame Buffer Bus Width (bits) & 256 \\ \hline 
			RAM Type & GDDR5 \\ \hline 
			Frame Buffer Bandwidth (GB/s) & 256,256 \\ \hline 
			Graphics Clock (MHz) & 1746,5 \\ \hline 
			Processor Clock (MHz) & 1746,5 \\ \hline 
			Memory Clock (MHz) & 4004 \\ \hline 
			SM Count & 15 \\ \hline 
			CUDA Cores & 1920 \\ \hline 
		\end{tabular}%

	\end{table}%
		
%\bibliography{./ref.bib}
%\bibliographystyle{ieeetr}

\end{document}